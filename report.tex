\documentclass{article}
\usepackage[14pt]{extsizes} 
\usepackage[utf8]{inputenc}
\usepackage[russian]{babel}
\usepackage{graphicx}
\graphicspath{{pictures/}}
\DeclareGraphicsExtensions{.pdf,.png,.jpg}
\usepackage{xcolor}
\usepackage{hyperref}
\usepackage{listings}
\usepackage{float}
\usepackage{indentfirst}
\usepackage[left=2.5cm, right=1.5cm, vmargin=2.5cm]{geometry}

\begin{document}

  \begin{center}
	   ГУАП\\
	   КАФЕДРА № 51
  \end{center}
  \begin{center}
\begin{tabular}{clcll}
& & & & \\
\multicolumn{1}{l}{ПРЕПОДАВАТЕЛЬ} & \multicolumn{1}{c}{} & & & \\ \cline{1-3}
\multicolumn{1}{|c|}{доцент, к.т.н.} & \multicolumn{1}{l|}{} & \multicolumn{1}{c|}{Линский Е.М.} & & \\ \cline{1-3}
\multicolumn{1}{|c|}{\begin{tabular}[c]{@{}c@{}}должность , уч. степень,\\ звание\end{tabular}} & \multicolumn{1}{c|}{подпись, дата} & \multicolumn{1}{c|}{инициалы, фамилия} &  &  \\ \cline{1-3}
\end{tabular}
\end{center}

\vspace{5cm}

  \begin{center}
	ОТЧЕТ О ЛАБОРАТОРНОЙ РАБОТЕ № 8\\
	СОЗДАНИЕ ПРОГРАММЫ НА ЯЗЫКЕ JAVA\\
	\vspace{1cm}
	по курсу: ТЕХНОЛОГИИ ПРОГРАММИРОВАНИЯ\\
  \end{center}
	
	\vspace{4cm}
	\begin{center}
\begin{tabular}{cccll}
& \multicolumn{1}{l}{} & & &  \\
\multicolumn{1}{l}{РАБОТУ ВЫПОЛНИЛ} & & & &  \\ \cline{1-4}
\multicolumn{1}{|c|}{СТУДЕНТ ГР. №} & \multicolumn{1}{c|}{5511} & \multicolumn{1}{c|}{} & \multicolumn{1}{c|}{Вдовенко А.} & \\ \cline{1-4}
\multicolumn{1}{|c|}{} & \multicolumn{1}{c|}{} & \multicolumn{1}{c|}{подпись, дата} & \multicolumn{1}{c|}{\begin{tabular}[c]{@{}c@{}}инициалы,\\ фамилия\end{tabular}} &  \\ \cline{1-4}
\end{tabular}
\end{center}
	\vspace{1cm}
  \begin{center}
	Санкт-Петербург 2017
  \end{center}
\thispagestyle{empty}
\newpage
	\textbf{Задание:}
	\\Реализовать класс ParallelMatrixProduct для многопоточного 
	умножения матриц UsualMatrix. В конструкторе класс получает 
	число потоков, которые будут использованы для перемножения 
	(число потоков может быть меньше, чем число строк у первой 
	матрицы).
	\\В функции main сравнить время перемножения больших случайных 
	матриц обычным и многопоточным способом. Получить текущее 
	время можно с помощью методов класса System.\\
	
	
	\textbf{Дополнительное задание:}
	\\ Сделать реализацию методов сортировки QuickSort и 
	ParallelQuickSort.
	\\ Сравнить в main две реализации с помощью 
	System.currentTimeMillis() (замерить время до вызова сортировки, 
	потом после вызова сортировки, вычесть). Выявить число потоков, 
	при котором метод ParallelQuickSort будет выполняться за 
	наименьшее время.\\
	
	
	\textbf{Инструкция:}
	\\При запуске программы используются все вышеописанные 
	методы.\\
	
	
	\textbf{Тестирование:}
	\begin{enumerate}
		\item Создаем два объекта UsualMatrix размерами 1000х1000. 
		Заполняем их случайными целыми числами в диапазоне от 
		0 до 100.
		\item Вычисляем произведение матриц. Замеряем время, которое 
		потребовалось для выполнения метода умножения. 
		Измеренное время: 16930.
		\item Вычисляем произведение матриц, используя 
		многопоточный метод умножения. Количество 
		используемых потоков равно 10. Замеряем время, которое 
		потребовалось для выполнения многопоточного метода 
		умножения. Измеренное время: 11465.
		\item Создаем два объекта ArrayList<Integer> и добавляем в них 
		100000 элементов случайных целых чисел в диапазоне от 0 
		до 100.
		\item Используем метод сортировки quickSort. Замеряем время, 
		которое потребовалось на выполнение метода. Измеренное 
		время: 78.
		\item Используем метод сортировки parallelQuickSort. 
		Количество используемых потоков равно 2. Замеряем 
		время, которое потребовалось на выполнение метода. 
		Измеренное время: 16.
	\end{enumerate}
\end{document}
